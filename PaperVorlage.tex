% Bei Arial 11pt, bei Times New Roman 12pt
\documentclass[11pt,a4paper]{scrreprt}

\usepackage[utf8]{inputenc}
\usepackage[german]{babel}
\usepackage[T1]{fontenc}
\usepackage{amsmath}
\usepackage{amsfonts}
\usepackage{amssymb}
\usepackage{graphicx}
\usepackage{lmodern}

% Arialähnliche Schrift
\usepackage{helvet}
\renewcommand{\familydefault}{\sfdefault}

% Um Subsubsections im Inhaltsverzeichnis anzuzeigen
\setcounter{tocdepth}{4}
\setcounter{secnumdepth}{4}

% Für Zeilenabstand
\usepackage[onehalfspacing]{setspace}
% Alternativ für komplettes Dokument:
% \usepackage[singlespacing]{setspace}
% \usepackage[onehalfspacing]{setspace}
% \usepackage[doublespacing]{setspace}
% Um im Text zu ändern: \singlespacing, \onehalfspacing oder \doublespacing
% Überschriften über mehrere Zeilen 1-fach
% Blockzitate 1-fach
% Beschriftung Tabellen und Abbildungen 1-fach
% Fußnoten 1-fach
% Listen 1-fach oder 1,5-fach
% Verweise 1-fach

% Times New Roman ähnliche Schrift
% \usepackage{mathptmx}
% \renewcommand{\familydefault}{\rmdefault}

% Bei gebundenen Arbeiten (Bachelor- oder Masterarbeiten) links auf 4cm
\usepackage[left=2cm,right=2cm,top=2cm,bottom=2cm]{geometry}

\author{Sebastian Wittmann}
\title{Rhein Waal Vorlage}

\usepackage{scrlayer-scrpage}
\pagestyle{scrheadings}
\clearpairofpagestyles
\chead{\thepage}

\usepackage{acronym}


\begin{document}

% Deckblatt Bachelorarbeit
% \begin{titlepage}
% \begin{center}
% \fontsize{16pt}{16pt}\selectfont Hochschule Rhein-Waal\\
% Fakultät Kommunikation und Umwelt\\
% Prof. Dr. Jane Public\\
% Prof. Dr. Martin Mustermann\\
% \vspace{4cm}
% \fontsize{22pt}{22pt}\selectfont\textbf{Titel}\\
% \fontsize{20pt}{20pt}\selectfont Untertitel\\
% \vspace*{12cm}
% \fontsize{16pt}{16pt}\selectfont Bachelorarbeit\\
% \vspace{2cm}
% vorgelegt von\\
% John Sample
% \end{center}
% \end{titlepage}






% Titelseite Bachelorarbeit
% \begin{titlepage}
% \begin{center}
% \fontsize{16pt}{16pt}\selectfont Hochschule Rhein-Waal\\
% Fakultät Kommunikation und Umwelt\\
% Prof. Dr. Jane Public\\
% Prof. Dr. Martin Mustermann\\
% \vspace{2cm}
% \fontsize{22pt}{22pt}\selectfont\textbf{Titel}\\
% \fontsize{20pt}{20pt}\selectfont Untertitel\\
% \vspace*{4cm}
% \fontsize{16pt}{16pt}\selectfont Bachelorarbeit\\
% im Studiengang\\
% Medien- und Kommunikationsinformatik\\
% zur Erlangung des akademischen Grades\\
% \vspace{1cm}
% Bachelor of Science\\
% \vspace{2cm}
% vorgelegt von\\
% John Sample\\
% Beispielstraße 18\\
% 12345 Beispielstadt\\
% \vspace{1cm}
% Matrikelnummer:\\
% 98765\\
% \vspace{1cm}
% Abgabedatum:\\
% Tag Monat Jahr
% \end{center}
% \end{titlepage}




% Titelseite Seminararbeit
\begin{titlepage}
\begin{center}
\fontsize{16pt}{16pt}\selectfont Hochschule Rhein-Waal\\
Fakultät Kommunikation und Umwelt\\
\vspace{2cm}
Seminararbeit\\
WS/SS 20xx/20xx\\
Modul: Modulname\\
\vspace{2cm}
\fontsize{22pt}{22pt}\selectfont\textbf{Titel}\\
\fontsize{20pt}{20pt}\selectfont Untertitel\\
\vspace{8cm}
John Sample\\
\vspace{2cm}
Matrikelnummer:\\
98765\\
\vspace{2cm}
Abgabetermin:\\
Tag Monat Jahr
\end{center}
\end{titlepage}

\pagenumbering{roman}
\setcounter{page}{2}


% Max 250 Zeichen und immer englisch
\chapter*{Abstract}
\thispagestyle{scrheadings}
Here goes the abstract

\chapter*{Vertraulichkeitsklausel}
\thispagestyle{scrheadings}
Die folgende [Semesterarbeit, Seminararbeit, Bachelorarbeit, Masterarbeit, …] enthält vertrauliche Daten und Informationen, die durch <Name der Drittpartei und Gesellschaftsform> bereitgestellt wurden. Die Daten und Informationen dürfen vor dem <Datum> oder ohne ausdrückliche Erlaubnis durch <Name der Drittpartei und Gesellschaftsform> weder in ihrer Gesamtheit noch in Auszügen preisgegeben, weitergeleitet oder in irgend-einer Weise öffentlich gemacht werden. Die [Semesterarbeit, Seminar-arbeit, Bachelorarbeit, Maserarbeit, …] wird den Prüfern ausschließlich zum Zwecke der Beurteilung der Arbeit zur Verfügung gestellt.

\tableofcontents
\thispagestyle{scrheadings}


\chapter*{Abkürzungsverzeichnis}
\thispagestyle{scrheadings}
\begin{acronym}
\acro{z.B.}[z.B.]{zum Beispiel}
\end{acronym}

\listoffigures
\thispagestyle{scrheadings}

\listoftables
\thispagestyle{scrheadings}






\chapter{Erstes Kapitel}
\pagenumbering{arabic}
\thispagestyle{scrheadings}
Hier steht Text

\section{Erstes Unterkapitel}
Hier muss auch Text stehen

\section{Zweiter Unterkapitel}
Hier auch Text
\begin{figure}[h]
Hier ist eine Abbildung
\caption{Abbildung Nummer eins}
\end{figure}

\subsection{Erstes Unterunterkapitel}
Diese Kapitel sind auch möglich. Subsubsection sollte vermieden werden.\\
\begin{table}[h]
\begin{tabular}{l l}
Spalte 1 & Spalte 2
\end{tabular}
\caption{Tabelle 1}
\end{table}

\subsubsection{Erstes Unterunterunterkapitel}
Möglich, sollte aber vermieden werden

\subsubsection{Zweites Unterunterunterkapitel}
Ich bin kein Text, ich putz' hier nur

\subsection{Zweites Unterunterkapitel}
Man kennt es

\section{Drittes Unterkapitel}
Und auch hier kommt was hin \footnote{Fußnoten dürfen auch verwendet werden}

\section{Viertel Unterkapitel}
''Zitate über 40 Zeichen sollten links und rechts 1 cm Platz haben''

\appendix
\chapter{Anhang 1}
Hier kann mehr Zeug hin

\chapter{Anhang 2}
Und hier auch nochmal was


\end{document}